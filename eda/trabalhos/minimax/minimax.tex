\documentclass[sigplan,screen]{acmart}

\usepackage[ruled]{algorithm2e}
\renewcommand{\algorithmcfname}{ALGORITHM}
%% Remove Permissions and ACM Reference Notes
\renewcommand\footnotetextcopyrightpermission[1]{}
\settopmatter{printacmref=false}
%%
%% \BibTeX command to typeset BibTeX logo in the docs
\AtBeginDocument{%
  \providecommand\BibTeX{{%
    \normalfont B\kern-0.5em{\scshape i\kern-0.25em b}\kern-0.8em\TeX}}}

\begin{document}

\title{Árvores Minimax}
\subtitle{DIMXXXX - Estruturas de Dados}

\author{Tiago Vinícius Remígio da Costa}
\email{vinicius.remigio@gmail.com}
\affiliation{
  \institution{DIMAp - Universidade Federal do Rio Grande do Norte}
  \city{Natal}
  \state{RN}
  \country{Brasil}
}

\begin{abstract}
  Árvores minimax tem sido usadas em conceitos de jogos de soma zero. 
  Neste trabalho veremos algumas implicações desta estrutura de dados, 
  assim como pseudo-código e exemplos de utilização em problemas reais.
\end{abstract}

\keywords{algoritmos, estruturas de dados, árvores}

\maketitle
\pagestyle{plain}

\section{Introdução}
ACM's consolidated article template, introduced in 2017, provides a
consistent \LaTeX\ style for use across ACM publications, and
incorporates accessibility and metadata-extraction functionality
necessary for future Digital Library endeavors. Numerous ACM and
SIG-specific {\itshape LaTeX} templates have been examined, and their unique
features incorporated into this single new template.

\section{Funcionamento}
As noted in the introduction, the ``\verb|acmart|'' document class can
be used to prepare many different kinds of documentation --- a
double-blind initial submission of a full-length technical paper, a
two-page SIGGRAPH Emerging Technologies abstract, a ``camera-ready''
journal article, a SIGCHI Extended Abstract, and more --- all by
selecting the appropriate {\itshape template style} and {\itshape
  template parameters}.

\subsection{Pseudo-código}

\begin{algorithm}
  \caption{Bytecode Generation Overview}
  \label{alg:generator}
  \SetKwProg{generate}{Function \emph{generate}}{}{end}
  
  Map store=new Map(obj, queue)\;
  \generate{Object pivot}{
       \ForAll{child $c$ in pivot}{
       \If{ $c$'s FieldContext is not set and $c$ is fusible}{
            generate($c$)\;
        }
       }
       build pivot's fieldContext $fc$\;
       EmitClassName\;
       EmitFields($fc$)\;
       EmitMethods($fc$)\;
  }
\end{algorithm}

\subsection{Template Parameters}

In addition to specifying the {\itshape template style} to be used in
formatting your work, there are a number of {\itshape template parameters}
which modify some part of the applied template style. A complete list
of these parameters can be found in the {\itshape \LaTeX\ User's Guide.}

Frequently-used parameters, or combinations of parameters, include:
\begin{itemize}
\item {\verb|anonymous,review|}: Suitable for a ``double-blind''
  conference submission. Anonymizes the work and includes line
  numbers. Use with the \verb|\acmSubmissionID| command to print the
  submission's unique ID on each page of the work.
\item{\verb|authorversion|}: Produces a version of the work suitable
  for posting by the author.
\item{\verb|screen|}: Produces colored hyperlinks.
\end{itemize}

This document uses the following string as the first command in the
source file:
\begin{verbatim}
\documentclass[sigplan,screen]{acmart}
\end{verbatim}


\section{Trade-offs}

The ``\verb|acmart|'' document class requires the use of the
``Libertine'' typeface family. Your \TeX\ installation should include
this set of packages. Please do not substitute other typefaces. The
``\verb|lmodern|'' and ``\verb|ltimes|'' packages should not be used,
as they will override the built-in typeface families.

\section{Aplicações}

\section{Conclusão}

\section{Citations and Bibliographies}

\bibliographystyle{ACM-Reference-Format}
\bibliography{sample-base}

\end{document}
\endinput
